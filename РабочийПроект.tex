\newsection
\section{Рабочий проект}
\subsection{Классы, используемые при разработке сайта}

Можно выделить следующий список классов и их методов, использованных при разработке web-приложения (таблица \ref{class:table}).

\begin{longtable}[l]{|p{3cm}|p{3cm}|p{4cm}|p{4cm}|}
\caption{Описание классов Bitrix, используемых в приложении\label{class:table}}\\
\hline Название класса & Модуль, к которому относится класс & Описание класса & Методы \\
\hline \centering 1 & \centering 2 & \centering 3 & 4\\
\endfirsthead
\caption*{Продолжение таблицы \ref{class:table}}\\
\hline \centering 1 & \centering 2 & \centering 3 & 4\\
\endhead
\hline CMain & Главный модуль & CMain – главный класс страницы web-приложения. После одного из этапов по загрузке страницы в сценарии становится доступным инициализированный системой объект данного класса с именем \$APPLICATION & void ShowTitle(string property\_code = «title», bool strip\_tags = true)
Выводит заголовок страницы
void SetTitle(string title)
Устанавливает заголовок страницы

void ShowCSS(bool external = true, bool XhtmlStyle = true)
Выводит таблицу стилей CSS страницы\\
\hline CFile & Главный модуль & CFile – Класс для работы с файлами и изображениями & array GetFileArray (int file\_id)
Метод возвращает массив, содержащий описание файла (путь к файлу, имя файла, размер) с идентификатором file\_id\\
  \hline
\end{longtable}

\subsection{Тестирование разработанного web-сайта}

На рисунке \ref{main:image} представлена главная страница сайта «Русатом – Аддитивные технологии».

\begin{figure}[H]
\center{\includegraphics[width=1\linewidth]{main1}}
\center{\includegraphics[width=1\linewidth]{main2}}
\center{\includegraphics[width=1\linewidth]{main3}}
\caption{Главная страница сайта «Русатом – Аддитивные технологии»}
\label{main:image}
\end{figure}

На рисунке \ref{menu:image} представлен динамический вывод заголовков, включающий в себя искомые фразы при поиске фраз.

\begin{figure}[H]
\center{\includegraphics[width=1\linewidth]{menu}}
\caption{Динамический вывод заголовков}
\label{menu:image}
\end{figure}

На рисунке \ref{enter:image} представлен ввод данных для публикации новости.

\begin{figure}[H]
\center{\includegraphics[width=1\linewidth]{enter}}
\caption{Ввод данных для публикации очень-очень длинной, интересной и полезной новости}
\label{enter:image}
\end{figure}
